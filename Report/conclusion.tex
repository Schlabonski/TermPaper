\section{Conclusion}
\label{sec:Conclusion}
Having now visited some of the experiments that Serge Haroche and David Wineland
realized, the similarities between their fields of research have become even more
clear. Theoretically single trapped  ions and atoms in cavities can be 
treated equivalently by applying the Jaynes-Cummings Hamiltonian and the method of Fock states
to describe either the photon number or the quantum number of motion. Also
experimentally we see that to observe the desired quantum mechanical effects both
of them put huge effort into the isolation of single quanta.

To this day Serge Haroche continues to do research in the field of cavity
quantum electrodynamics, some of his current interests being non-local quantum
states and the quantum engineering of Rydberg states, at École Normale
Supérieure in Paris. Also David Wineland continues to be the leader of the ion
storage group at NIST working on high precision time standards and quantum
information processing, this being another interesting parallel between the two.

Especially in association with prestigious prizes, the work of the laureates is
often depicted as the sole performance of extraordinary individuals. At least in
the case of Serge Haroche and David Wineland this is not true, as they both
pointed out in their Nobel lectures~\cite{wineland2012nobel, haroche2012nobel}.
Both of them had important colleagues some of which spent more than forty years
in the laboratories with them and contributed important ideas to their work.
Both of them have been lucky to work along with great scientists of current and preceding
generations from their very first days of research. Further it is important to
realize what role technological progress played. Without the invention of
tunable lasers, pulsed lasers, ion traps or superconducting materials, none of
the experiments would have been possible. To put it in the words of Serge
Haroche:
\begin{quote}
  ``On the professional side, I have had the luck to embark in a field - atomic
  physics and quantum optics - which has undergone  fantastic  developments
  over this period of time, improving by many orders of magnitudes the
  sensitivity of experiments and the precision of measurements. Thanks to
  advances in laser technology, new domains have been explored, in ultra-low
  temperature physics or in the study of ultrafast phenomena for instance, that
  we could not even imagine at the time I was working for my PhD. I did not work
  myself in many of these fields, but I witnessed these developments as a member
  of a very active and imaginative community of physicists, sharing the
  excitement and the bewilderment brought about by all these spectacular
  advances.''~\cite{shbio}
\end{quote}

This said we end the term paper with the remark that both of the laureates have left an
impressive lifework that was and is definitely worth the study.
