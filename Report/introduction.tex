\section{Introduction}
The Nobel prize in physics 2012 was awarded jointly to the French physicist
Serge Haroche and the American physicist David Wineland.
According to the Nobel jury they received it ``for ground-breaking experimental
methods that enable measuring and manipulation of individual quantum systems''.
Serge Haroche spent most of his life researching at the École Normale Supérieure
in Paris. There he created photon traps, i.e. cavities, in which microwave
photons could be stored up to half a second~\cite{haroche2007QuantumJumps}. To
probe the wave function of the photons, he sent in groups of or single atoms to
see how they would interact with the light. In this way he unveiled many of the
interesting effects of the quantum nature of light. David Wineland spent most of
his scientific life on the other side of the Atlantic, at the National Institute
of Standards and Technology in Boulder, Colorado. His drive was to invent better
methods to slow down and
isolate ions as this allowed frequency measurements of
higher and higher precision. To achieve this, he used sophisticated setups
combining ion traps with high precision lasers. 

But if one of the laureates dedicated his research to photons and the other one
to ions, why did they share the Nobel prize? The answer to this question lies in
the congruence of their methods with respect to quantum mechanics. Haroche was
probing photons with atoms, Wineland examined the energy levels of atoms using
photons, but both were dealing with the same underlying effects of quantum
mechanics. This can be easily seen when one leaves aside their specific physical
objects of study (i.e. photons and ions) and inspects which abstract topics
they were dealing with. Both of them published papers about ``Schrödinger Cat
States''~\cite{brune1992manipulation, monroe1996schrodinger}, bringing to
physical experiment one of the most fundamental thought experiments of early
quantum mechanics. Both conducted experiments on many particle entanglement
~\cite{rauschenbeutel2000step, sackett2000experimental}. Both found ways to
engineer explicitly  non-classical states in their systems~\cite{meekhof1996generation,
deleglise2008reconstruction}. This list could be continued, but the message is
clear: each of them in their own fields found ways to answer fundamental issues
of quantum mechanics experimentally and on their way introduced countless new
methods to the physics community.

The outline of this term paper is as follows: first we will theoretically derive
the phenomenon of Rabi oscillations from the Jaynes-Cummings Hamiltonian, because
both, Haroche and Wineland, used it as a central method in their later
experiments (Sec.~\ref{sec:Theory}) . Then we will have a look at  each of the
laureate's early life and inspect some of their central experiments in order to
get an impression of their experimental creativity and innovation
(Sec.~\ref{sec:Haroche}~and~\ref{sec:Wineland}). A short conclusion will then be
given in Sec.~\ref{sec:Conclusion}.
