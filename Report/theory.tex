\section{Theory}

\subsection{Jaynes-Cummings Hamiltonian}
Rabi cycles are a phenomenon that occurs when an atom interacts with light.
Both, Serge Haroche and David Wineland, investigated and made use of the
shifts in energylevels and relative phases of states that appear when a laser
shines in on an atom.  In
order give a useful and complete description of Raby cycles, one has to take into
account that the energy levels of the atom and of the photon field are
quantized. The first ``fully quantized'' approach for the case of coherent light
in a cavity and a two level system was given by Edwin Jaynes and Fred Cummings
in 1963 \cite{jaynes1963comparison}. To describe the system (following
\cite{gerry2005introductory}) 
we introduce the total Hamiltonian, consisting of the Hamiltonian for the 
 quantized photon field for a single mode, the Hamiltonian of the two level
 system consisting of a ground state $\ket{g}$ and an excited state $\ket{e}$ and the
 interaction Hamiltonian
\begin{align}
  \label{eq:JCM_H_tot}
  {\hat {H}}={\hat {H}}_{{{\text{field}}}}+{\hat {H}}_{{{\text{atom}}}}+{\hat
  {H}}_{{{\text{int}}}},
\end{align}
where, neglecting the zero point energies of field and atom,
\begin{align}
  \label{eq:JCM_H_field}
  \hat{H}_{\text{field}} &= \hbar \omega_{\text{f}}\, \hat{a}^\dagger
   \hat{a}\\
  \label{eq:JCM_H_atom}
  \hat{H}_{\text{atom}} &= \frac{\hbar \omega_a}{2} \, \hat{\sigma_3} \\
  \label{eq:JCM_H_int}
  \hat{H}_{\text{int}} &= \hbar \omega_{\text{int}} \left( \hat{\sigma}_+ +
\hat{\sigma}_-\right)\left(\hat{a} + \hat{a}^\dagger\right).
\end{align}
Here $\hat{a}^\dagger$ and $\hat{a}$ are the creation and annihilation operators
that appear in the quantization of the electromagnetic field\footnote{To solve
the free field equation of the elmg. field one usually uses the Fourier ansatz
$$A^\mu(x) = \int \text{d}\tilde{k} \sum_{\lambda=0}^3 \left[ a_\lambda(\vec{k})
\epsilon^\mu_\lambda(k)\text{e}^{-ikx} +
a^\dagger_\lambda(\vec{k})\epsilon^\mu_\lambda(k)^* \text{e}^{+ikx}\right], $$
where $\lambda$ goes over all possible polarizations. When demanding that
$A^\mu$ and its conjugate field $\pi^\nu$ obey the canonical quantization
relation, the resulting commutator relations for $a_\lambda$ and
$a_\lambda^\dagger$ allow the interpretation as creation and annihilation
operators.}, the Pauli matrix $\hat{\sigma_3} = \ket{e}\bra{e} - \ket{g}\bra{g}$
acts as the inversion operator of the atom, and the associated Pauli
matrices $\hat{\sigma}_+ = \ket{e}\bra{g}$ and $\hat{\sigma}_- = \ket{g}\bra{e}$
project any state $\ket{g}$ ($\ket{e}$) to $\ket{e}$ ($\ket{g}$) respectively
and can thus be seen as atomic transition operators. It is now useful to look at
the time evolution of the terms in the total Hamiltonian \eqref{eq:JCM_H_tot}.
To do so we switch to the interaction picture, using the Hamiltonian without
interaction $\hat{H}_0 = \hat{H}_{\text{atom}} + \hat{H}_{\text{field}}$ to
describe the time evolution of the operators. For the  annihilation operator we
have 
\begin{align}
  \label{eq:a_time_ev}
  \hat{a}(t) &= e^{i\hat{H_0}t/\hbar}\, \hat{a}(0)\, e^{-i\hat{H_0}t/\hbar}
  \nonumber \\
  \intertext{using the Baker-Campbell-Hausdorff formula, we obtain}
  &= \hat{a}(0) + \frac{it}{\hbar}\left[\hat{H}_0, \hat{a}\right] +
  \frac{1}{2!}\left(\frac{it}{\hbar}\right)^2\left[\hat{H}_0, \left[\hat{H}_0,
  \hat{a}\right] \right] + \dots \nonumber
  \intertext{The first commutator is $\left[\hat{H}_0, \hat{a}\right] = -\hbar \omega_f
  \hat{a}$ as $\hat{a}$ commutes with $\hat{H}_{\text{atom}}$ and
  $\left[\hat{a}^\dagger, \hat{a}\right]=-1$. The nested commutators will thus only add higher orders of $\omega_f$. The
time evolution then becomes}
&= \hat{a}(0) \left(1 - it\omega_f + (it\omega_f)^2 + \dots  \right) \nonumber
\\
&= \hat{a}(0) e^{-i\omega_f t}.\,\footnotemark
\end{align}
\footnotetext{By looking at the preceeding footnote
and considering that $kx$ is a scalar product of four-vectors but the
integration over $\text{d}\tilde{k}$ is only over the three spatial components,
we see that the time evolution behaviour is already built-in in this ansatz.}Correspondingly we obtain for the other operators
\begin{align}
  \label{eq:ops_time_ev}
  \hat{a}^\dagger(t) &= \hat{a}^\dagger(0)e^{i\omega_ft}\\
  \hat{\sigma}_\pm(t) &= \hat{\sigma}_\pm(0)e^{\pm i\omega_a t}
\end{align}
and therefore the full interaction Hamiltonian becomes
\begin{align}
  \label{eq:H_int_time_ev}
  \hat{H}_{\text{int}}(t) = \hbar \omega_{\text{int}} \left( \hat{\sigma}_+ \hat{a}
  \, e^{i(\omega_a-\omega_f)t} \right& + \hat{\sigma}_+ \hat{a}^\dagger\,
    e^{i(\omega_a+\omega_f)t}\nonumber\\ & \left\quad + \hat{\sigma}_-\hat{a}\, e^{-i(\omega_a+\omega_f)t}
+ \hat{\sigma}_-\hat{a}^\dagger \, e^{-i(\omega_a-\omega_f)t}\right).
\end{align}
Assuming that the photon frequency $\omega_f$ is close to the transition
frequency of the atom $\omega_a$, i.e. $\left| \omega_a-\omega_f \right|
\ll \omega_a+\omega_f $, the rotating wave approximation can be applied. This
means that all terms in $\hat{H}_{\text{int}}$ that oscillate with
$\omega_a+\omega_f$ are
neglected. Doing this and transforming back to the Schrödinger picture leaves us
with the total Hamiltonian
\begin{align}
  \label{eq:H_tot_rot_wave}
  \hat{H}_{\text{tot}}= \hbar \omega_f \hat{a}^\dagger \hat{a} + \frac{\hbar\omega_a}{2}\hat{\sigma}_3
  + \hbar \omega_{\text{int}}\left( \hat{\sigma}_+\hat{a} +
  \hat{\sigma}_-\hat{a}^\dagger  \right).
\end{align}
The last two terms can be seen as processes in which the atom aborbs or emmits
one photon from the field while respectively changing its internal energy state.

\subsection{Rabi Oscillations}
Having found a Hamiltonian that describes the interaction of an atom with light,
it is now interesting to investigate how this interaction influences the
dynamics of the system. The original states of atom and fiel will no longer be
eigenstates of the system and thus undergo a continuous oscillation, called Rabi
oscillation. Bot, Serge Haroche and David Wineland have made experimental use of
this effect in order to prepare and manipulate states. To describe the dynamics
it is first useful to split the Hamiltonian in \eqref{eq:H_tot_rot_wave} into
two parts, namely
\begin{align}
  \label{eq:H_I_H_II}
  \hat{H}_I &= \hbar\omega_f\underbrace{\left( \hat{a}^\dagger\hat{a} +
  \ket{e}\bra{e}\right)}_{\text{excitation number }\hat{N}_e} + \hbar \left( \frac{\omega_a}{2} -
\omega_f)\right)\underbrace{\left(\ket{e}\bra{e} +
\ket{g}\bra{g}\right)}_{\text{e$^-$ number projector }\hat{P}_e} \\
\hat{H}_{II} &= -\hbar \underbrace{\left(\omega_a -
\omega_f\right)\ket{g}\bra{g}}_{\equiv\Delta} +
\hbar\omega_{\text{int}}\left( \hat{\sigma}_+\hat{a} +
  \hat{\sigma}_-\hat{a}^\dagger  \right).
\end{align}
The first part $\hat{H}_I$ commutes with $\hat{H}_{\text{tot}}$ thus it is
conserved over time and any interesting dynamics of the system are described by
the second part. Let us now consider a state
\begin{align}
  \label{eq:psi_t}
  \ket{\psi(t)} = C_1(t)\ket{e}\ket{n} + C_2(t)\ket{g}\ket{n+1} 
\end{align}
with initial conditions $C_1(0) = 1$ and $C_2(0)=0$. The time evolution is
described by the time dependend Schrödinger equation
$i\hbar\frac{\text{d}}{\text{d}t}\ket{\psi(t)}
= \hat{H}_{II}\ket{\psi(t)}$. In the resonant case ($\Delta=0$) this can be
exactly solved and yields
\begin{align}
  \label{eq:psi_t_solution}
  C_1(t) &= \cos\left(\omega_{\text{int}}\sqrt{n+1}\,t\right) \\
  C_2(t) &= -i\sin\left(\omega_\text{int} \sqrt{n+1} \,t\right)
\end{align}

\subsection{Dressed States}

